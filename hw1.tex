%\documentclass[12pt]{article}
%\documentclass[10pt,conference]{IEEEtran}
%\documentclass[12pt,draftcls,onecolumn]{IEEEtran}
\documentclass[journal,12pt,draftclsnofoot,onecolumn]{IEEEtran}
%\documentclass[conference]{IEEEtran}
%\documentclass[10pt,journal,onecolumn]{IEEEtran}
%\documentclass[10pt,journal,twocolumn]{IEEEtran}
%\documentclass[12pt,journal,onecolumn]{IEEEtran}
%\documentclass[twocolumn,11pt]{article}
%\documentclass[12pt]{article}
%\textwidth=6.6in

%\usepackage{amsmath,amssymb,epsf,latexsym,makeidx}
\usepackage[cm]{fullpage}
%\usepackage{setspace} % this is with \doublespacing, must be put after\begin{document}
%\usepackage{graphicx}
%\usepackage{caption,subcaption} % for subfigures

%\setlength{\topmargin}{-0.6in}
%\setlength{\oddsidemargin}{-0.05in}
%\setlength{\textwidth}{6.6in}
%\setlength{\textheight}{9.0625in}




\renewcommand{\baselinestretch}{1}
\newtheorem{theorem}{Theorem}[section]
\newtheorem{conjecture}{Conjecture}[section]
\newtheorem{prop}[theorem]{Proposition}
\newtheorem{lemma}[theorem]{Lemma}
\newtheorem{definition}[theorem]{Definition}
\newtheorem{corollary}[theorem]{Corollary}
\newtheorem{algorithm}[theorem]{Algorithm}
\newtheorem{defi}{Definition}
%\newenvironment{proof}{\indent {\bf Proof.}}{\hfill$\bf  \Box$\bigskip}
%\newenvironment{proofof}{\indent {\bf Proof of}}{\hfill$\bf  \Box$\bigskip}

\newcommand{\post}[2]{
\centering \leavevmode
%  (Sichao's version:   \includegraphics[width=#2cm,height=#3cm]{#1.eps}
 \includegraphics[width=#2cm]{#1}
 }

%\evensidemargin 0.6in
%\oddsidemargin -0.2in
%\topmargin 0in
%\textwidth 6.8in
%\headheight 0.0in
%\headsep 0.25in
%\textheight 8.9in
%\footskip 0.25in
%\evensidemargin 0.0in
%\oddsidemargin 0.0in
%\topmargin -0.05in
%\textwidth 6.5in
%\headheight 0.0in
%\headsep 0.0in
%\textheight 8.9in
%\footskip 0.25in


\newcommand{\mat}[2][rrrrrrrrrrrrrrrrrrrrrrrrrrrrrrrrrrr]{\left[
                    \begin{array}{#1}
                    #2\\
                    \end{array}
                    \right]}

\newcommand{\dat}[2][rrrrrrrrrrrrrrrrrrrrrrrrrrrrrrrrrrr]{\left|
                    \begin{array}{#1}
                    #2\\
                    \end{array}
                    \right|}


\title{homework 1, CS 421}
\author{
\authorblockN{Ji Zhu, jizhu1@illinois.edu}\\
\authorblockA{Department of Electrical and Computer Engineering \\
 and the Coordinated Science Laboratory  \\
 University of Illinois at Urbana-Champaign}
}

\begin{document}
%\doublespacing
\maketitle

Apply $\rho_{i}$ to denote the environment at the $i$th point.
\begin{enumerate}
\item $\rho_{1} = \emptyset$
\item $\rho_{2} = \{a \rightarrow 10, x\rightarrow 21\}$
\item $\rho_{3} = \{ \{f \rightarrow <x\rightarrow y \rightarrow x - y + a, \rho_{2}>\} + \rho_{2}\} $
\item $ \rho_{4} =\{ \{b \rightarrow 21, a\rightarrow 5, x\rightarrow 21\}+\{f \rightarrow <x\rightarrow y \rightarrow x - y + a, \rho_{2}>\} \}$, $\rho_{3}$ is put into the stack.
\item $\rho_{3}$ is retrieved from the stack, $\rho_{5} = \rho_{3}$
\item $\rho_{6} = \{ \{f \rightarrow <x\rightarrow y \rightarrow x - y + a, \rho_{2}>\} + \{a\rightarrow 10, x\rightarrow 1\} \}$
\item $ \rho_{7} = \{ \{h\rightarrow <y\rightarrow f~2 ~20, \rho_{6}>\}  + \rho_{6}  \}  $
\item $\rho_{7}$ is put into the stack,
$$\rho_{8} =\{ \{h\rightarrow <y \rightarrow f~(y+4), \rho_{7}>\} + \{f\rightarrow <x\rightarrow y\rightarrow x * y, \rho'>\} + \rho''\} $$
where $\rho' = \{ \{h\rightarrow <y \rightarrow f~(y+4), \rho_{7}>\} + \rho_{6} \}$,  $\rho'' =\{a\rightarrow 10, x\rightarrow 1\} $
\item $\rho_{7}$ is retrieved from the stack, and $\rho_{9} = \rho_{7}$
\end{enumerate}

%\section{}

%\begin{figure} 
%\begin{subfigure}[b]{0.5\textwidth}
%              \post{trees4.pdf}{7}
%                \caption{...}
%                \label{fig:tree4}
%\end{subfigure}
%\caption{...} \label{fig:trees}
%\end{figure}


\bibliographystyle{IEEEtran}
%\bibliography{prelim.bib}

\end{document}
