%\documentclass[12pt]{article}
%\documentclass[10pt,conference]{IEEEtran}
%\documentclass[12pt,draftcls,onecolumn]{IEEEtran}
\documentclass[journal,12pt,draftclsnofoot,onecolumn]{IEEEtran}
%\documentclass[conference]{IEEEtran}
%\documentclass[10pt,journal,onecolumn]{IEEEtran}
%\documentclass[10pt,journal,twocolumn]{IEEEtran}
%\documentclass[12pt,journal,onecolumn]{IEEEtran}
%\documentclass[twocolumn,11pt]{article}
%\documentclass[12pt]{article}
%\textwidth=6.6in

%\usepackage{amsmath,amssymb,epsf,latexsym,makeidx}
\usepackage[cm]{fullpage}
\usepackage{amsfonts,amsmath}
%\usepackage{setspace} % this is with \doublespacing, must be put after\begin{document}
\usepackage{graphicx}
\usepackage{url}
%\usepackage{caption,subcaption} % for subfigures

%\setlength{\topmargin}{-0.6in}
%\setlength{\oddsidemargin}{-0.05in}
%\setlength{\textwidth}{6.6in}
%\setlength{\textheight}{9.0625in}




\renewcommand{\baselinestretch}{1}
\newtheorem{theorem}{Theorem}[section]
\newtheorem{conjecture}{Conjecture}[section]
\newtheorem{prop}[theorem]{Proposition}
\newtheorem{lemma}[theorem]{Lemma}
\newtheorem{definition}[theorem]{Definition}
\newtheorem{corollary}[theorem]{Corollary}
\newtheorem{algorithm}[theorem]{Algorithm}
\newtheorem{defi}{Definition}
%\newenvironment{proof}{\indent {\bf Proof.}}{\hfill$\bf  \Box$\bigskip}
%\newenvironment{proofof}{\indent {\bf Proof of}}{\hfill$\bf  \Box$\bigskip}

\newenvironment{problem}[1]{{\bf Problem. #1}}{\indent \hfill\bigskip}
\newenvironment{solution}{{\bf Solution: }}{\hfill\bigskip}

\newcommand{\prob}[3]{
{\bf Problem. #1}
{\indent #2}

{\bf Solution: } #3\bigskip

}

\newcommand{\vect}[2]{
\left\langle#1,#2\right\rangle
}


\newcommand{\post}[2]{
\centering \leavevmode
%  (Sichao's version:   \includegraphics[width=#2cm,height=#3cm]{#1.eps}
 \includegraphics[width=#2cm]{#1}
 }

%\evensidemargin 0.6in
%\oddsidemargin -0.2in
%\topmargin 0in
%\textwidth 6.8in
%\headheight 0.0in
%\headsep 0.25in
%\textheight 8.9in
%\footskip 0.25in
%\evensidemargin 0.0in
%\oddsidemargin 0.0in
%\topmargin -0.05in
%\textwidth 6.5in
%\headheight 0.0in
%\headsep 0.0in
%\textheight 8.9in
%\footskip 0.25in


\newcommand{\mat}[2][rrrrrrrrrrrrrrrrrrrrrrrrrrrrrrrrrrr]{\left[
                    \begin{array}{#1}
                    #2\\
                    \end{array}
                    \right]}

\newcommand{\dat}[2][rrrrrrrrrrrrrrrrrrrrrrrrrrrrrrrrrrr]{\left|
                    \begin{array}{#1}
                    #2\\
                    \end{array}
                    \right|}

\newcommand{\ben}{\begin{eqnarray*}}
\newcommand{\een}{\end{eqnarray*}}
\newcommand{\non}{\nonumber}
\newcommand{\benu}{\begin{enumerate}}
\newcommand{\ennu}{\end{enumerate}}

\title{Homework 2 for cs 421, Fall 2012}
\author{
\authorblockN{Ji Zhu, jizhu1}
}





\begin{document}
%\doublespacing
\maketitle

\begin{enumerate}
\item (1). $\rho_{1} = \{x\rightarrow 5\}$, $\rho_{2} = \{x\rightarrow 5, y\rightarrow 3, z\rightarrow 8\}$, the environment at point 1 is
$$
a_{1} = \{\text{plus\_x}\rightarrow \vect{y\rightarrow x+y}{ \rho_{1} }\} + \rho_{2}
$$

(2). $\rho_{3} = \{x\rightarrow 5, y\rightarrow -5, z\rightarrow 8\}$, the environment at point 2 is
$$
a_{2}=\{ \text{sub\_z}\rightarrow \vect{x\rightarrow y-z}{a_{1}} \}+
\{\text{plus\_x}\rightarrow \vect{y\rightarrow x+y}{ \rho_{1} }\} + \rho_{3}
$$

(3). The environment is
$$
a_{3} = \{\text{f\_z}\rightarrow \vect{x\rightarrow val}{a_{2}}\} +
\{ \text{sub\_z}\rightarrow \vect{x\rightarrow y-z}{a_{1}} \}+
\{\text{plus\_x}\rightarrow \vect{y\rightarrow x+y}{ \rho_{1} }\} + \rho_{3}
$$

\item The evaluation process is as follows:
\begin{itemize}
\item The value of $y$ is given to $x$: $x\rightarrow y\rightarrow -5$.
\item $x=-5$ is given to plus\_x, in the function of plus\_x, $y\rightarrow x\rightarrow -5$, $x\rightarrow 5$ from $\rho_{1}$, and result $0$ is returned.  
\item The result $0$ is compared with $z\rightarrow 8$ in function f\_z, the result is true.
\item $x = -5$ is given to sub\_z, in function sub\_z, $x\rightarrow -5$, $y\rightarrow 3, z\rightarrow 8$ from $a_{1}$, and result $3-8 = -5$ is returned. 
\item Back in function f\_z, the returned value $-5$ is returned and outputted.
\end{itemize}
\end{enumerate}


%\section{}

%\begin{figure} 
%\begin{subfigure}[b]{0.5\textwidth}
%              \post{trees4.pdf}{7}
%                \caption{...}
%                \label{fig:tree4}
%\end{subfigure}
%\caption{...} \label{fig:trees}
%\end{figure}


\bibliographystyle{IEEEtran}
%\bibliography{prelim.bib}

\end{document}
